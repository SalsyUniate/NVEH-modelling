\documentclass[a4paper, french, 12pt, titlepage]{report}
\usepackage[latin1, utf8]{inputenc}
\usepackage[T1]{fontenc}
\usepackage[french]{babel}
\usepackage{graphicx,  amsmath, amssymb}
\usepackage[margin = 1in]{geometry}
\graphicspath{{./images/}}


\title{Rapport de stage \\ Modélisation numérique d’oscillateurs non-linéaires pour
la récupération d’énergie vibratoire}
\author{LÉGLISE Cloé}

\begin{document}

\maketitle

\section{Introduction}

Dans le cadre de mes études en école d'ingénieurs, j'ai dû effectuer un stage type assistant ingénieur durant la quatrième année de mon cycle ingénieur. 

\section{Sujet de stage}

Dans le monde de la recherche, le domaine de récupération d'énergie vibratoire a pour enjeu de compléter, voire remplacer l'usage des batteries classiques, dont la durée de vie est limitée.  

\section{Résumé des papiers}

\section{Cahier des charges}

\section{Mise en place}
utilisation de VScode, julia (vs python), git.... 

\section{Travail effecué}

\section{Ce qu'il reste à faire}

\section{Conclusion}



\section*{Bibliographie}

Saint-Martin, C., Morel, A., Charleux, L., Roux, E., Benhemou, A., & Badel, A. (2022). Power expectation as a unified metric for the evaluation of vibration energy harvesters. Mechanical Systems and Signal Processing, 181, 109482.

Huguet, T. (2018). Vers une meilleure exploitation des dispositifs de récupération d’énergie vibratoire bistables: Analyse et utilisation de comportements originaux pour améliorer la bande passante (Doctoral dissertation, Université de Lyon).

Liu, W. (2014). Conception d'un dispositif de récupération d'énergie vibratoire large bande (Doctoral dissertation, Grenoble).


\end{document}