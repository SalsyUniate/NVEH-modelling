\documentclass[a4paper, french, 12pt, titlepage]{report}
\usepackage[latin1, utf8]{inputenc}
\usepackage[T1]{fontenc}
\usepackage[french]{babel}
\usepackage{graphicx,  amsmath, amssymb}
\usepackage[margin = 1in]{geometry}
\graphicspath{{./images/}}


\title{Rapport de stage \\ Modélisation numérique d’oscillateurs non-linéaires pour
la récupération d’énergie vibratoire}
\author{LÉGLISE Cloé}

\begin{document}

\maketitle

\section{Introduction}

Dans le cadre de mes études en école d'ingénieurs, j'ai dû effectuer un stage type assistant ingénieur durant la quatrième année de mon cycle ingénieur. 

\section{Sujet de stage}

Dans le monde de la recherche, le domaine de récupération d'énergie vibratoire a pour enjeu de compléter, voire remplacer l'usage des batteries classiques, dont la durée de vie est limitée.  

\section{Résumé des papiers}

L'utilisation de récupérateurs d'énergie permet de récolter de l'énergie vibratoire. Il en existe de plusieurs sortes, par exemple les récupérateurs d'énergie piézoélectriques linéaires, qui peuvent amplifier les vibrations s'ils sont excités à leur fréquence naturelle. Ceci dit, ces récupérateurs d'énergie ne peuvent récupérer une puissance importante qu'au sein d'une bande passante très étroite. Les récupérateurs d'énergie vibratoire non linéaires ont une bande passante bien plus large, bien que la complexité de leurs comportement peuvent les rendre difficiles à analyser. 

Les récupérateurs d'énergie non linéaires multi-stables peuvent présenter plusieurs cycles limites stables, appelés orbites, qui dépendent des conditions initiales. Certaines de ces orbites, appelées orbites hautes, correspondent à un mouvement oscillatoire entre deux positions stables de la masse inertielle. D'autres, appelées orbites basses, correspondent à une situation à la masse inertielle oscille faiblement autour d'une seule position stable. Les orbites hautes sont beaucoup plus intéressantes pour la récupération d'énergie, parce qu'elles représentent une puissance plus importante. Il est donc intéressant de mettre en place des stratégies de saut d'orbites, permettant à l'oscillateur de passer d'une orbite faible à une orbite basse. 

Pour effectuer un saut d'orbite, il faut nécessairement donner de l'énergie à l'oscillateur. Il faut donc trouver une méthode la moins coûteuse en énergie possible pour que la manipulation soit rentable. Pour un oscillateur bistable de Duffing, cinq paramètres jouent un rôle important dans son comportement : la masse M, la raideur k, la valeur des positions stables $x_0$, la longueur L et l'amortissement µ. Notons que le niveau de flambement, donné par $\frac{x_0}{L}$, sera indirectement modifié par la modification d'autres paramètres. Ainsi, en touchant l'un de ces paramètres, on peut augmenter l'énergie potentielle du système et ainsi le faire passer d'une orbite basse à une orbite haute. 


\section{Cahier des charges}

\section{Mise en place}
utilisation de VScode, julia (vs python), git.... 

\section{Travail effecué}

\section{Ce qu'il reste à faire}

\section{Conclusion}



\section*{Bibliographie}

Saint-Martin, C., Morel, A., Charleux, L., Roux, E., Benhemou, A., & Badel, A. (2022). Power expectation as a unified metric for the evaluation of vibration energy harvesters. Mechanical Systems and Signal Processing, 181, 109482.

Huguet, T. (2018). Vers une meilleure exploitation des dispositifs de récupération d’énergie vibratoire bistables: Analyse et utilisation de comportements originaux pour améliorer la bande passante (Doctoral dissertation, Université de Lyon).

Liu, W. (2014). Conception d'un dispositif de récupération d'énergie vibratoire large bande (Doctoral dissertation, Grenoble).

	
\end{document}

