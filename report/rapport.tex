\documentclass[a4paper, french, 12pt, titlepage]{report}
\usepackage[latin1, utf8]{inputenc}
\usepackage[T1]{fontenc}
\usepackage[french]{babel}
\usepackage{graphicx,  amsmath, amssymb}
\usepackage[margin = 1in]{geometry}
\graphicspath{{./images/}}


\title{Rapport de stage \\ Modélisation numérique d’oscillateurs non-linéaires pour
la récupération d’énergie vibratoire}
\author{LÉGLISE Cloé}

\begin{document}

\maketitle

\section*{Résumé des papiers}





\section*{Bibliographie}

Saint-Martin, C., Morel, A., Charleux, L., Roux, E., Benhemou, A., & Badel, A. (2022). Power expectation as a unified metric for the evaluation of vibration energy harvesters. Mechanical Systems and Signal Processing, 181, 109482.

Huguet, T. (2018). Vers une meilleure exploitation des dispositifs de récupération d’énergie vibratoire bistables: Analyse et utilisation de comportements originaux pour améliorer la bande passante (Doctoral dissertation, Université de Lyon).

Liu, W. (2014). Conception d'un dispositif de récupération d'énergie vibratoire large bande (Doctoral dissertation, Grenoble).


\end{document}